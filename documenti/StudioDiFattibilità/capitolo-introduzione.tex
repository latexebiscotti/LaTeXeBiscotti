\section{Introduzione}

\subsection{Scopo del documento}
Questo documento descrive valutazioni e motivazioni che hanno portato, da parte del gruppo \GroupName{}, la decisone di realizzare il progetto \ProjectName{}. 
Vengono pertanto elencati pregi e criticità del capitolato scelto, illustrando poi, gli aspetti decisivi che ci hanno portato ad escludere le restanti proposte. 

\subsection{Capitolato scelto}
Capitolato: C3 - \ProjectName{}: un motore per l'analisi predittiva in ambiente \glossario{Internet of Things}\\
Proponente: \Proponente{} \\
Committente: \Committente{} \\

\subsection{Scopo del Prodotto}
\ScopoDelProdotto

\subsection{Glossario}
Per evitare tutte le possibili ambiguità sul linguaggio utilizzato e per massimizzare la comprensione da parte di tutti dei documenti, della terminologia specifica e di quella di dominio, degli acronimi e di tutte quelle parole che necessitano chiarimento (contraddistinte da una G pedice), viene redatto un \textit{Glossario}, consultabile nel documento \Glossario.

\subsection{Riferimenti}
\subsubsection{Normativi}
\begin{itemize}
	\item \textbf{\NormeDiProgetto};
	\item\textbf{ Capitolato d'appalto C3:} \ProjectName{}: un motore per l'analisi predittiva in ambiente \glossario{Internet of Things}: \url{http://www.math.unipd.it/~tullio/IS-1/2015/Progetto/C3.pdf};
	\item \textbf{Vincoli sull'organigramma del gruppo e sull'offerta tecnico-economica:} \\ \url{http://www.math.unipd.it/~tullio/IS-1/2015/Progetto/PD01b.html}.
\end{itemize}

\subsubsection{Informativi}
\begin{itemize}
	\item \textbf{Slide dell'insegnamento Ingegneria del Software modulo A:}
	\begin{itemize}
		\item Ciclo di vita del Software;
		\item Gestione di Progetto.
	\end{itemize}
	\url{http://www.math.unipd.it/~tullio/IS-1/2015/}
	\item \textbf{\textit{Software Engineering} - Ian Sommerville - 9th Edition (2011):}
	\begin{itemize}
		\item Part 4: Software Management.
	\end{itemize} 
\end{itemize}