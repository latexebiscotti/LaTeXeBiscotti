\section{capitolato C3}

\subsection{Valutazione costi e benefici}
Il Capitolato volge la sua attenzione alla creazione di un Software generico per un'analisi di grandi quantità di dati provenienti da macchinari non precedentemente definiti. Questo Software deve essere in grado di individuare i dati sensibili dalla base dati prodotta dal sistema cui verrà applicato, riportarli all'utente attraverso un front-end applicativo e mantenere un'attività di apprendimento di quelli che potrebbero essere cambi o aggiunte di nuovi dati rilevanti. Il Software che si andrebbe a creare avrebbe un ottimo inserimento all'interno del mercato attuale vista la crescente espansione del campo delle IoT. Ciò che lo rende veramente competitivo è il fatto di non essere specializzato per un singolo macchinario, quindi creato ad hoc, ma applicabile in potenza a qualsiasi ambiente che richieda analisi reattiva di dati ricevuti da macchine collegate ad una rete.