\section{Altri capitolati}

\subsection{C1 - Actorbase: a NoSQL DB based on the Actor model}
\subsubsection{Valutazione generale}
Analizzando questo capitolato, il gruppo ha gradito la proposta del committente, in quanto prende in esame due aspetti molto interessati:
\begin{itemize}
	\item \textbf{Database \glossario{NoSQL}}: attualmente infatti si scelgono sempre pi\`u questi tipi di database, andando a sostituire i pi\`u ormai obsoleti e inadeguati database relazionali, nel caso in cui lo scopo dell'applicativo sia un'interazione con una grande mole di dati.
	A questo proposito la proposta del committente rappresentava un'ottima possibilit\`a, nello studio di questi nuovi modelli di database;
	\item \textbf{\glossario{Actor Model}}: questo modello gi\`a implementato da nella libreria \glossario{Akka}, rappresenta sicuramente un ottimo modello di studio nella programmazione concorrente.
\end{itemize}
Valutati questi come principali aspetti positivi, che rendono il capitolato un'ottimo progetto a scopo didattico, il gruppo ha per\`o deciso di scartarlo non riuscendo ad immaginarne un'utilizzo pratico ed immediato, che veniva proposto maggiormente da altri capitolati.
Inoltre si \`e valutato come, escluso il modello ad attori, al gruppo del tutto nuovo, e allo sviluppo di un database \glossario{NoSQL}, non ci fossero nuove tecnologie o modelli su cui il team di lavoro avrebbe dovuto lavorare e confrontarsi rendendo il capitolato meno appetibile.  

\subsection{Potenziali Criticità}

\begin{itemize}
	\item \textbf{Variazione dei requisiti:} nel capitolato in questione è specificato che i requisiti richiesti possono essere modificati durante la realizzazione del sistema. Questo ci preoccupava molto, soprattutto visto il rischio di dovere fare un passo indietro con il lavoro svolto per poter inserire eventuali requisiti nuovi o modificare quelli già esistenti.
\end{itemize}

\subsection{C2 - CLIPS: Communication \& Localisation with Indoor Positioning System}
\subsubsection{Valutazione generale}
Questo capitolato affronta un tema innovativo e di grande interesse non solo per quanto riguarda in nostro gruppo di lavoro ma anche per tutto l'ambiente informatico e tecnologico, ovvero l'\glossario{Internet of things}.
Per interagire all'interno dell'\glossario{IoT} in questo progetto sono stati presentati i \glossario{Beacons}, che per\`o hanno causato all'interno del gruppo incertezza riguardante la scelta di tale capitolato.
Sicuramente lo scenario si presenta innovativo riguardando lo sviluppo della tecnologia \glossario{BLE}, basata sulla versione 4.0 del \glossario{Bluetooth}, tuttavia come detto i \glossario{Beacons}, rappresentando una tecnologia giovane ha esposto ancora alcuni difetti quali:
\begin{itemize}
	\item La frequenza per aggiornare il rilevamento di dispositivi \`e programmabile, ma utilizzando una frequenza media utile, a quello che \`e di fatto un utilizzo da parte di utenti in movimento, si va ad intaccare in modo considerevole la durata della batteria;
	\item La portata del segnale \`e legata strettamente alla potenza con cui \`e stato programmato il \glossario{Beacons} e nel caso in cui il dispositivo con cui comunica debba inviare dei dati, riferiti alla sua posizione, sono spesso non del tutto precisi e non particolarmente inviati reattivamente;
	\item Il posizionamento dei \glossario{Beacons}, deve avvenire ad un'altezza minima di 2.50 metri d'altezza per evitare la perdita di segnale dovuta all'interferenza con corpi liquidi e persone.
\end{itemize}
Questi aspetti assieme alla perplessit\`a sull'effettiva necessit\`a di strumenti del genere gi\`a in alcuni casi sostituiti da altre tecnologie pi\`u affidabili, precise e mature, ci ha fatto decidere di non prendere parte a questo progetto.

\subsection{Potenziali Criticità}

\begin{itemize}
	\item \textbf{Innovazione:} trovare un campo dove inserire questo tipo di tecnologia che sia innovativo non è un compito semplice;
\end{itemize}

\begin{itemize}
	\item \textbf{Beacons:} l'eccessiva instabilità dei dispositivi beacons potrebbe compromettere lo svolgimento del progetto e incrementare molto il carico di lavoro necessario;
\end{itemize}

\begin{itemize}
	\item \textbf{Sperimentazione Pratica:} la richiesta di effettuare almeno 2 prove per la sperimentazione pratica, con relative descrizioni e documentazioni, rischia di allungare sensibilmente i tempi per la realizzazione del progetto;
\end{itemize}



\subsection{C4- MaaS: MongoDB as an admin Service}
\subsubsection{Valutazione generale}



\subsection{C5 - Quizzipedia: software per la gestione di questionari}
\subsubsection{Valutazione generale}
In questo progetto le tecnologie utilizzate sono molte ed interessanti, per altro non del tutto sconosciute ad alcuni dei membri del gruppo.
Tecnologie quali:
 \begin{itemize}
	\item \textbf{\glossario{SQL} o \glossario{NoSQL}};
	\item\textbf{\glossario{Tomcat} o \glossario{Node.js}};
	\item \textbf{\glossario{HTML5}};
	\item \textbf{\glossario{CSS3}};
	\item \textbf{\glossario{Javascrip}}.
\end{itemize}
Tuttavia nonostante queste, la richiesta di un sistema che preveda un archivio di domande e di test che interrogato tale archivio fornisca all'utente questionari specifici per l'argomento scelto, non ha raccolto nel gruppo grande interesse, escludendolo onde evitare un approccio gi\`a errato inizialmente per un progetto che richiedecess coinvolgimento e partecipazione a tuttotondo.

\subsection{Potenziali Criticità}

\begin{itemize}
	\item \textbf{Sottovalutazione del progetto:} essendo la proposta la più semplice sulla carta, almeno per quanti riguarda i requisiti minimi, il rischio maggiore con questo capitolato è di sottovalutare l'entità del lavoro necessario e di non riuscire quindi a rimanere nei tempi prefissati.
\end{itemize}


\subsection{C6 - SiVoDiM: Sintesi Vocale per Dispositivi Mobili}
\subsubsection{Valutazione generale}
Questo capitolato tratta un argomento senza dubbio che si distingue dai restanti capitolati, affrontare un tema come la sintesi vocale \`e infatti fonte di curiosit\`a per alcuni membri del gruppo, ma non rappresenta un'interesse unanime.
Nonostante si tratti di utilizzare un motore di sintesi \glossario{opensource}, con tutti i vantaggi del genere, come \virgolette{Flexible and Adaptive Text To Speech} (FA-TTS), l'applicazione richiesta che, avrebbe dovuto girare su dispositivi mobili, altro aspetto di indubbio valore, per come si presentano i mercati tecnologici odierni, tratta una tecnologia troppo di nicchia che ci ha pertanto scoraggiato nella scelta di questo capitolato.

\subsection{Potenziali Criticità}

\begin{itemize}
	\item \textbf{Innovazione:} trovare un campo dove inserire questo tipo di tecnologia che sia innovativo non è un compito semplice;
\end{itemize}

\begin{itemize}
	\item \textbf{Startup:} il basso numero di personale a disposizione del proponente rischia di rendere difficoltosa l'interazione con esso. Se nelle fasi cruciali del progetto per un problema interno alla startup non si riuscisse a comunicare in tempo con il proponente per togliere eventuali dubbi, l'avanzamento potrebbe subire seri rallentamenti e il tempo di realizzazione aumentare conseguentemente.
\end{itemize}
