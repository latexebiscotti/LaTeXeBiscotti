\input{../../modello/layout}
%%%%%%%%%%%%%%
%  COSTANTI  %
%%%%%%%%%%%%%%

% In questa prima parte vanno definite le 'costanti' utilizzate da due o più documenti.

% Meglio non mettere gli \emph dentro le costanti, in certi casi creano problemi
\newcommand{\GroupName}{LaTeXeBiscotti}
\newcommand{\GroupEmail}{latexebiscotti@gmail.com}
\newcommand{\ProjectName}{UMAP}
\newcommand{\ProjectNameSub}{un motore per l'analisi predittiva in ambiente Internet of Things}
\newcommand{\ProjectVersion}{v1.0.0}

\newcommand{\Proponente}{Zero12}
\newcommand{\Committente}{Prof. Tullio Vardanega \\ Prof. Riccardo Cardin}
\newcommand{\Responsabile}{}

% La versione dei documenti deve essere definita qui in global, perchè serve anche agli altri documenti
\newcommand{\VersioneG}{1.0.2}
\newcommand{\VersionePQ}{0.0.0}
\newcommand{\VersioneNP}{1.0.8}
\newcommand{\VersionePP}{1.1.0}
\newcommand{\VersioneAR}{0.0.0}
\newcommand{\VersioneSF}{0.0.0}
\newcommand{\VersioneST}{0.0.0}
\newcommand{\VersioneMA}{0.0.0}
\newcommand{\VersioneMS}{0.0.0}
\newcommand{\VersioneMU}{0.0.0}
\newcommand{\VersioneDP}{0.0.0}
% Il verbale non ha versionamento.
% Lasciare vuoto, non mettere trattini o puntini
% Non sono permessi numeri nel nome di un comando :(
\newcommand{\VersioneVprimo}{} 
\newcommand{\VersioneVsecondo}{}

% Quando serve riferirsi a ``Nome del Documento + ultima versione x.y.z'' usiamo queste costanti:
\newcommand{\Glossario}{\emph{Glossario v\VersioneG{}}}
\newcommand{\PianoDiQualifica}{\emph{Piano di Qualifica v\VersionePQ{}}}
\newcommand{\NormeDiProgetto}{\emph{Norme di Progetto v\VersioneNP{}}}
\newcommand{\PianoDiProgetto}{\emph{Piano di Progetto v\VersionePP{}}}
\newcommand{\StudioDiFattibilita}{\emph{Studio di Fattibilità v\VersioneSF{}}}
\newcommand{\AnalisiDeiRequisiti}{\emph{Analisi dei Requisiti v\VersioneAR{}}}
\newcommand{\SpecificaTecnica}{\emph{Specifica Tecnica v\VersioneST{}}}
\newcommand{\ManualeAdmin}{\emph{Manuale Admin v\VersioneMA{}}}
\newcommand{\ManualeSviluppatore}{\emph{Manuale Sviluppatore v\VersioneMS{}}}
\newcommand{\ManualeUtente}{\emph{Manuale Utente v\VersioneMU{}}}
\newcommand{\DefinizioneDiProdotto}{\emph{Definizione di Prodotto v\VersioneDP{}}}

\newcommand{\ScopoDelProdotto}{
	Lo scopo del progetto è la realizzazione di un \glossario{algoritmo predittivo} in ambiente \glossario{Internet of Things} in grado analizzare i dati provenienti da “oggetti”, inseriti in diversi contesti, e fornire delle previsioni su possibili guasti, interazioni con nuovi utenti ed identificare dei pattern di comportamento degli utenti per prevedere le azioni degli stessi su altri oggetti o altri contesti. 
}

%%%%%%%%%%%%%%
%  FUNZIONI  %
%%%%%%%%%%%%%%

% In questa seconda parte vanno definite le 'funzioni' utilizzate da due o più documenti.

% Comando per usare comodamente le virgolette
\newcommand{\virgolette}[1]{``#1''}

% Serve a dare la giusta formattazione alle parole presenti nel glossario
% il nome del comando \glossary è già usato da LaTeX
\newcommand{\glossario}[1]{\textit{#1\ped{\ped{G}}}}

\newcommand*\flextt{%
  \fontdimen2\font=0.4em% interword space
  \fontdimen3\font=0.6em% interword stretch
  \fontdimen4\font=0.1em% interword shrink
  \fontdimen7\font=0.1em% extra space
  \hyphenchar\font=`\:
}

% Serve a dare la giusta formattazione al codice inline
\newcommand{\code}[1]{\flextt{\texttt{#1}}}

% Serve a dare la giusta formattazione a tutte le path presenti nei documenti
\newcommand{\file}[1]{\flextt{\texttt{#1}}}

% Permette di andare a capo all'interno di una cella in una tabella
\newcommand{\multiLineCell}[2][c]{\begin{tabular}[#1]{@{}l@{}}#2\end{tabular}}

% Genera automaticamente la pagina di copertina
\newcommand{\makeFrontPage}{
  % Declare new goemetry for the title page only.
  \newgeometry{top=3.5cm}
  
  \begin{titlepage}
  \begin{center}

  \begin{center}
  \includegraphics[width=10cm]{../../modello/cookie.png}
  \end{center}
  
  \vspace{1cm}

  \begin{Huge}
  \textbf{\DocTitle{}}
  \end{Huge}
  
  \textbf{\emph{Gruppo \GroupName{} \, \texttwelveudash{} \, Progetto \ProjectName{}}}
  
  \vspace{11pt}

  \bgroup
  \def\arraystretch{1.3}
  \begin{tabular}{ r|l }
    \multicolumn{2}{c}{\textbf{Informazioni sul documento}} \\
    \hline
		% differenzia a seconda che \DocVersion{} stampi testo o no
		\setbox0=\hbox{\DocVersion{}\unskip}\ifdim\wd0=0pt
			% nulla (non ho trovato come togliere l'a capo)
			\\
		\else
			\textbf{Versione} & \DocVersion{} \\
		\fi
    \textbf{Redazione} & \multiLineCell[t]{\DocRedazione{}} \\
    \textbf{Verifica} & \multiLineCell[t]{\DocVerifica{}} \\
    \textbf{Approvazione} & \multiLineCell[t]{\DocApprovazione{}} \\
    \textbf{Uso} & \DocUso{} \\
    \textbf{Distribuzione} & \multiLineCell[t]{\DocDistribuzione{}} \\
  \end{tabular}
  \egroup

  \vspace{22pt}

  \textbf{Descrizione} \\
  \DocDescription{}

  \end{center}
  \end{titlepage}
  
  % Ends the declared geometry for the titlepage
  \restoregeometry
}
%%%%%%%%%%%%%%
%  COSTANTI  %
%%%%%%%%%%%%%%

% In questa prima parte vanno definite le 'costanti' utilizzate soltanto da questo documento.
% Devono iniziare con una lettera maiuscola per distinguersi dalle funzioni.

\newcommand{\DocTitle}{Analisi dei Tequisiti�}
\newcommand{\DocVersion}{\Versione{AR}}

\newcommand{\DocRedazione}{Filippo Todescato}
\newcommand{\DocVerifica}{}
\newcommand{\DocApprovazione}{}

\newcommand{\DocUso}{Esterno}
\newcommand{\DocDistribuzione}{
	\Committente{} \\
	Gruppo \GroupName{}
}

% La descrizione del documento
\newcommand{\DocDescription}{
 Questo documento descrive l'analisi dei requisiti e i casi d'uso�del gruppo \GroupName{} relativo al progetto \ProjectName{}.
 }

%%%%%%%%%%%%%%
%  FUNZIONI  %
%%%%%%%%%%%%%%

% In questa seconda parte vanno definite le 'funzioni' utilizzate soltanto da questo documento.


\begin{document}

\makeFrontPage

\section*{Diario delle Modifiche}


\bgroup
\begin{longtable}{|P{1.8cm}|P{2.2cm}|P{3cm}|P{6cm}|}
 \hline \textbf{Versione} & \textbf{Data} & \textbf{Persone coinvolte} & \textbf{Descrizione} \\

 % IN ORDINE DALLA MODIFICA PIÙ RECENTE ALLA PIÙ VECCHIA
 \hline 1.0.9 & 2015-12-30 & Andrea Barcaro \linebreak (Amministratore) & Aggiunta Software scrittura documenti LaTeX. \\
 
 \hline 1.0.8 & 2015-12-24 & Andrea Barcaro \linebreak (Amministratore) & Modifica diagrammi di attività. \\
 
 \hline 1.0.7 & 2015-12-23 & Andrea Barcaro \linebreak (Amministratore) & Correzioni ortografiche, aggiunta software per le presentazioni. \\
 
 \hline 1.0.6 & 2015-12-18 & Andrea Barcaro \linebreak (Amministratore) & Terminata sezione \virgolette{Ambiente di Lavoro}, stesusa sezione \virgolette{Procedure a Supporto dei Processi}, aggiunta sottosezione \virgolette{Repository}.
 \\
 
 \hline 1.0.5 & 2015-12-17 & Andrea Barcaro \linebreak (Amministratore) & Continua stesura struttura sezione \virgolette{Ambiente di Lavoro}, aggiunta diagrammi di attività.
 \\
 
 \hline 1.0.4 & 2015-12-16 & Andrea Barcaro \linebreak (Amministratore) & Stesura struttura sezione \virgolette{Ambiente di Lavoro}. \\
 
 \hline 1.0.3 & 2015-12-15 & Andrea Barcaro \linebreak (Amministratore) & Modifica sezione “Documenti”, riorganizzazione capitoli 2 e 3 in unico capitolo \virgolette{Collaborazione}. \\
 
 \hline 1.0.2 & 2015-12-14 & Andrea Barcaro \linebreak (Amministratore) & Stesura sezione “Documenti”. \\
  
 \hline 1.0.1 & 2015-12-13 & Andrea Barcaro \linebreak (Amministratore) & Stesura sezioni “Introduzione”, “Comunicazioni”, “Riunioni”. \\

 \hline 1.0.0 & 2015-12-13 & Andrea Barcaro \linebreak (Amministratore) & Stesura indice delle sezioni. \\

 \hline
\end{longtable}
\egroup


\clearpage
\tableofcontents

\letteraGlossario{A}
\definizione{Actor Model}
Modello matematico di esecuzione concorrente di un programma nel quale le primitive di elaborazione concorrente sono individuate negli attori.

\definizione{Akka}
Libreria \glossario{open source} che semplifica la costruzione di applicazioni concorrenti e distribuite per \glossario{JVM}.

\definizione{Algoritmo Predittivo}
Algoritmo che cerca di \virgolette{fare previsioni} sugli eventi futuri in base a dati raccolti in eventi passati.

\definizione{Amazon}
Societ\`a di commercio elettronico statunitense con sede a Seattle.

\definizione{Amazon web service}
 Raccolta di servizi web remoti offerti da \glossario{Amazon} operanti da 11 regioni geografiche distinte.

\definizione{Apache}
Organizzazione no-profit americana a sostegno di progetti software.

\definizione{API}
Acronimo di application programming interface, rappresentano le librerie software disponibili per un certo linguaggio di programmazione.
\clearpage

\letteraGlossario{B}
\definizione{Beacons}
Piccoli dispositivi che, attraverso la tecnologia \glossario{BLE}, sono in grado di trasmettere informazioni a smartphone e tablet.

\definizione{BLE}
Acronimo di Bluetooth Low Energy \`e una tecnologia ti tipo \glossario{Bluetooth} a basso consumo energetico.

\definizione{Bluetooth}
Standard di trasmissione per reti personali senza fili, sicuro ed economico che utilizza una frequenza radio a corto raggio.

\definizione{Bug}
Identifica un errore nella scrittura di un programma software.

\definizione{Bug Tracking}
Applicativo software utile al team di sviluppo di un progetto per tenere traccia delle segnalazioni di \glossario{bug} trovati nel proprio prodotto.
 
\definizione{Bytecode}
Linguaggio intermedio tra linguaggio di programmazione e linguaggio macchina che riduce l'indipendenza dall'hardware.
\clearpage

\letteraGlossario{C}
\definizione{Case}
Si accompagna agli aggettivi \textit{lower} o \textit{upper}, sta ad indicare rispettivamente se una lettera è minuscola o maiuscola.

\definizione{Caso d'uso}
Funzionalità di un prodotto software, o tecnica usata nei processi di ingegneria del software per effettuare in maniera esaustiva e non ambigua la raccolta dei requisiti, al fine di produrre software di qualità.

\definizione{Commit}
Parlando di \glossario{controllo di versione}, un commit si effettua quando si copiano le modifiche fatte su file locali nella cartella del \glossario{repository}.

\definizione{Controllo di Versione}
Gestione di versioni multiple di un insieme di informazioni, siano questi documenti testuali o parti di un programma software.

\definizione{CSS}
Acronimo di Cascading Style Sheets, ovvero fogli di stile, \`e un linguaggio usato nella formattazione di pagine web.

\definizione{CSS3}
Specifiche riguardati i fogli di stile \glossario{CSS}, costituite da sezioni separate dette moduli, con differenti stati di avanzamento e stabilit\`a. Alcuni di questi moduli sono stati pubblicati formalmente come \glossario{W3C Recommendation}, nel novembre 2014.
\clearpage

\letteraGlossario{D}
\definizione{Design Pattern}
Concetto che può essere definito come una soluzione progettuale generale ad un problema ricorrente. Si tratta di una descrizione o modello logico da applicare per la risoluzione di un problema che può presentarsi in diverse situazioni durante le fasi di progettazione e sviluppo del software, ancor prima della definizione dell'algoritmo risolutivo della parte computazionale.

\definizione{Debugging}
Attività che consiste nell'individuazione da parte del programmatore della porzione di software affetta da errore (\glossario{bug}) rilevata nei software a seguito dell'utilizzo del programma.
\clearpage

\letteraGlossario{F}
\definizione{Framework}
Architettura logica di supporto (spesso un'implementazione logica di un particolare \glossario{design pattern}) su cui un software può essere progettato e realizzato, spesso facilitandone lo sviluppo da parte del team di sviluppatori.
\clearpage

\letteraGlossario{G}
\definizione{Gantt}
Ideatore del \textit{diagramma di Gantt}, strumento usato nelle attività di \glossario{project management} per tenere sotto controllo tutte le attività correlate al progetto in una determinata fascia temporale.

\definizione{Git}
Sistema software di \glossario{controllo di versione} distribuito.

\definizione{GitHub}
GitHub è un servizio web di hosting per lo sviluppo di progetti software, che usa il sistema di controllo di versione \glossario{Git}.
\clearpage

\letteraGlossario{H}
\definizione{HTML}
Linguaggio di markup per la strutturazione di pagine web.

\definizione{HTML5}
Linguaggio \glossario{HTML} pubblicato come \glossario{W3C Recommendation} da ottobre 2014.

\definizione{HTTP}
Acronimo di HyperText Transfer Protocol \`e un protocollo utilizzato come sistema per la trasmissione di informazioni sul web. Le specifiche del protocollo sono gestite dal World Wide Web Consortium (W3C).

\definizione{HTTPS}
Acronimo di  HyperText Transfer Protocol over Secure Socket Layer \`e l'applicazione della crittografia asimmetrica al protocollo \glossario{HTTP}, per garantire trasferimenti di dati nel web evitando attacchi di tipo \glossario{man in the middle}.
\clearpage

\letteraGlossario{I}
\definizione{IDE}
Acronimo per Integrated Development Environment, ovvero un ambiente di sviluppo integrato per la realizzazione di programmi per sistemi informatici.

\definizione{Instant Messaging}
Categoria di sistemi di comunicazione in tempo reale in rete, tipicamente Internet o una rete locale, che permette ai suoi utilizzatori lo scambio di brevi messaggi.

\definizione{Internet of Things}
Neologismo riferito all'estensione di Internet al mondo degli oggetti e dei luoghi concreti.

\definizione{IoT}
Acronimo di \glossario{Internet of Things}.

\definizione{ISO}
Abbreviazione per International Organization for Standardization, è la più importante organizzazione a livello mondiale per la definizione di norme tecniche.
\clearpage

\letteraGlossario{J}
\definizione{Java}
linguaggio di programmazione orientato agli oggetti, progettato per essere indipendente dalla piattaforma di esecuzione, utilizzando l'implementazione di un processore virtuale detto \glossario{JVM}.

\definizione{Javascript}
Linguaggio di scripting orientato agli oggetti e agli eventi, utilizzato nella programmazione Web lato client per la creazione di effetti dinamici n siti e applicazioni web.

\definizione{JSON}
Acronimo di JavaScript Object Notation, \`e un formato usato nell''interscambio di dati fra applicazioni client-server.

\definizione{JVM}
Acronimo di Java Virtual Machine \`e il componente della piattaforma Java che esegue i programmi tradotti in \glossario{bytecode} dopo una prima compilazione.
\clearpage

\letteraGlossario{M}
\definizione{Machine learning}
Rappresenta un'area fondamentale nell'intelligenza artificiale e realizzando algoritmi e sistemi basti su osservazioni di dati per la sintesi di nuova conoscenza.

\definizione{Man in the middle}
Tipo di attacco crittografico nel quale l'attaccante \`e in grado di leggere, inserire o modificare a piacere, messaggi tra due parti comunicanti tra di loro.

\definizione{Milestone}
Importante traguardo intermedio nello svolgimento del progetto. Molto spesso è rappresentata da eventi, cioè da attività con durata zero o di un giorno, e viene evidenziata in maniera diversa dalle altre attività nell'ambito dei documenti di progetto. Può essere intesa anche come una particolare configurazione di item relativi al progetto.

\definizione{MongoDB}
Database non relazionale classificato come \glossario{NoSQL}, orientato ai documenti si tratta di software libero e \glossario{open source}.

\definizione{MQTT}
Acronimo di MQ Telemetry Transport (MQTT) \`e un protocollo di messaggistica posizionato in cima a TCP/IP, per situazioni a basso impatto e con banda limitata, che interagisce con un message broker responsabile della distribuzione dei messaggi ai client destinatari.
\clearpage

\letteraGlossario{N}
\definizione{Node.js}
\glossario{Framework} per lo sviluppo di applicazioni server-side di Javascript

\definizione{NoSQL}
Acronimo di Not Only SQL rappresentano tutti cui software che non utilizzano il modello relazionale, definendo le basi di date costruite in questo modo come memorizzazioni strutturate.
\clearpage

\letteraGlossario{O}
\definizione{Open source}
Accostato ad un software sta ad indicare che il codice sorgente dello stesso \`e pubblico, favorendone lo studio, le modifiche ed estensioni da parte di programmatori indipendenti.
\clearpage

\letteraGlossario{P}
\definizione{Package}
Collezione di classi e interfacce correlate.

\definizione{Play Framework}
\glossario{Framework} \glossario{open source} scritto in \glossario{Java} e \glossario{Scala}, ad elevata produttivit\`a che integra i componenti e le \glossario{API} necessarie per un moderno sviluppo delle applicazioni web.

\definizione{PNG}
Sigla per Portable Network Graphics, ovvero un formato file per immagini.

\definizione{Project Management}
Insieme di attività svolte tipicamene da una figura dedicata e specializzata detta project manager, volte all'analisi, alla progettazione, alla pianificazione e alla realizzazione degli obiettivi di un progetto, gestendolo in tutte le sue caratteristiche e fasi evolutive, nel rispetto di precisi vincoli (tempi, costi, risorse, scopi, qualità).
\clearpage

\letteraGlossario{R}
\definizione{Repository}
Luogo di memorizzazione dei file, spesso situato in un server remoto.

\definizione{Revert}
In ambito di \glossario{controllo di versione}, è l'abbandono di uno o più cambiamenti recenti in favore di un ritorno ad una precedente versione di un documento o di parti di software.
\clearpage

\letteraGlossario{S}
\definizione{Scala}
linguaggio di programmazione di tipo general-purpose multi-paradigma studiato per integrare le caratteristiche e funzionalit\`a dei linguaggi orientati agli oggetti e dei linguaggi funzionali.

\definizione{Skype}
Software proprietario freeware di \glossario{instant messaging} e VoIP. Con esso sono possibili le videochiamate e lo scambio di messaggi testuali o di file.

\definizione{SQL}
Acronimo di Structured Query Language \`e un linguaggio standardizzato per database che utilizzano il modello relazionale.

\definizione{SVG}
Acronimo di Scalable Vector Graphics, indica una tecnologia in grado di visualizzare oggetti di grafica vettoriale e, pertanto, di gestire immagini scalabili dimensionalmente.
\clearpage

\letteraGlossario{T}
\definizione{Telegram}
Servizio di \glossario{instant messaging} multipiattaforma, usato per inviare messaggi tramite Internet.

\definizione{Tomcat}
Server Web \glossario{open source} sviluppato dalla \glossario{Apache}. 

\definizione{Twitter Boostrap}
Raccolta di strumenti per la creazione di siti web, contenente modelli basati su \glossario{HTML}, \glossario{CSS} e \glossario{Javascript}.
\clearpage

\letteraGlossario{U}
\definizione{UML}
Acronimo per Unified Modeling Language, è un linguaggio visuale di modellazione e specifica basato sul paradigma object-oriented.

\definizione{UTF-8}
UTF-8, ovvero Unicode Transformation Format, 8 bit, è una codifica dei caratteri Unicode in sequenze di lunghezza variabile di byte.
\clearpage

\letteraGlossario{W}
\definizione{Way of Working}
La prassi, il modo di fare regolato da norme che segue un team o un'azienda nella produzione di prodotti software.

\definizione{W3C}
Sintesi di World Wide Web Consortium \`e la principale organizzazione per gli standard del World Wide Web.

\definizione{W3C Recommendation}
Standard formalmente dichiarati da parte del \glossario{W3C}.
\clearpage

\end{document}
