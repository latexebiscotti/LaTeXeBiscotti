\section{Analisi dei rischi}
Al fine di ottimizzare l'avanzamento delle attività di progetto, in modo da non compromettere le tempistiche o la qualità dei processi, è stata eseguita un'approfondita analisi dei rischi che prevede i seguenti passi:
\begin{itemize}
\item \textbf{Identificazione:} identificazione dei possibili rischi inerenti a:
	\begin{itemize}
	\item \textbf{Progetto:} influenzano la pianificazione, strumenti e risorse;
	\item \textbf{Prodotto:} influenzano la qualità o le prestazioni del software e la conformità rispetto alle aspettative del committente;
	\item \textbf{Mercato:} riguardanti i costi e la concorrenza.
	\end{itemize}
\item \textbf{Analisi:} considerazione, per ogni rischio individuato, della probabilità di occorrenza e valutazione delle conseguenze sul progetto;
\item \textbf{Pianificazione:} identificazione di strategie per evitare, minimizzare e gestire ognuno dei rischi precedentemente individuati;
\item \textbf{Controllo:} monitoraggio e valutazione regolare di ogni rischio per verificare il modificarsi della probabilità o degli effetti sul progetto.
\end{itemize}
La descrizione di ciascun rischio prevede uno schema fisso così composto da: nome, descrizione, probabilità di occorrenza, grado di pericolosità, strategie per rilevare il rischio e contromisure. 

\subsection{Rischi tecnologici}

\subsection{Rischi sul personale}
\subsubsection{Indisponibilità dei componenti del gruppo}
\begin{enumerate}
\item \textbf{Descrizione:} ciascun componente del gruppo può ammalarsi o avere impegni personali e necessità proprie che gli impediscono di lavorare al progetto in determinati momenti;
\item \textbf{Probabilità di occorrenza:} media;
\item \textbf{Grado di pericolosità:} medio;
\item \textbf{Strategie di rilevazione del rischio:} ogni componente è tenuto a comunicare al \textit{Responsabile di Progetto}, con adeguato anticipo, l'eventuale presenza di impegni che potrebbero ostacolare la sua collaborazione al progetto. L'utilizzo del calendario e del diagramma di Gantt, generati automaticamente da \glossario{\textit{Teamwork}}, aiutano ad avere una visione generale delle indisponibilità dei componenti; 
\item \textbf{Contromisure:} a seguito di una notifica da parte di un componente, il \textit{Respondabile di Progetto} si occuperà di suddividere il carico di lavoro tra gli altri componenti del gruppo e tale effetto non sarà troppo grave in quanto gli altri membri ne saranno a conoscenza in anticipo rispetto alla data di conclusione dell'attività. Nel caso invece di malattia improvvisa le contromisure potrebbero gravare di più sugli altri membri in quanto la riassegnazione probabilmente prevederà la suddivisione del lavoro su attività imminenti.
\end{enumerate}

\subsubsection{Problemi tra componenti del gruppo}
\begin{enumerate}
\item \textbf{Descrizione:} il gruppo di progetto è formato da individui eterogenei, probabilmente con principi diversi ed è per tutti la prima esperienza di collaborazione in un progetto di così grandi dimensioni. Questi fattori potrebbero causare incomprensioni tra di essi con conseguenti problemi di collaborazione e appesantimento del carico di lavoro;
\item \textbf{Probabilità di occorrenza:} bassa;
\item \textbf{Grado di pericolosità:} alto;
\item \textbf{Strategie di rilevazione del rischio:} la collaborazione e le interazioni tra i componenti del gruppo permettono al \textit{Responsabile di Progetto} di verificare la nascita di problematiche interpersonali;
\item \textbf{Contromisure:} è compito del \textit{Responsabile} fare da mediatore tra gli individui causa di contrasti. Se la situazione dovesse aggravarsi sempre più esso potrà prevedere una pianificazione che minimizzi il contatto tra questi membri.
\end{enumerate}

\subsubsection{Inesperienze del gruppo}
\begin{enumerate}
\item \textbf{Descrizione:} lo sviluppo del progetto richiede capacità di pianificazione e analisi che i componenti non possiedono a causa dell'inesperienza. Necessario è inoltre l'utilizzo di prodotti software sconosciuti alla maggior parte dei membri e alcune conoscenze richiedono del tempo prima di essere apprese;
\item \textbf{Probabilità di occorrenza:} alta;
\item \textbf{Grado di pericolosità:} alto;
\item \textbf{Strategie di rilevazione del rischio:} appena si presenta la necessità di utilizzare un nuovo strumento questo viene notificato ai membri del gruppo come descritto nelle \NormeDiProgetto{} in §2.1.2. In tal modo ognuno potrà cercare il materiale necessario a studiare la base teorica;
\item \textbf{Contromisure:} ogni componente del gruppo si impegna a studiare autonomamente dal materiale trovato. In caso di strumenti particolarmente ostici sono possibili incontri di studio collaborativo tra tutti o parte dei membri del gruppo.
\end{enumerate}

\subsection{Rischi organizzativi}
\begin{enumerate}
\item \textbf{Descrizione:}
\item \textbf{Probabilità di occorrenza:}
\item \textbf{Grado di pericolosità:}
\item \textbf{Strategie di rilevazione del rischio:}
\item \textbf{Contromisure:}
\end{enumerate}
\subsection{Rischi sugli strumenti software}

\subsection{Rischi sugli strumenti hardware}

\subsection{Rischi sui requisiti}

\subsection{Rischi sulle stime}


