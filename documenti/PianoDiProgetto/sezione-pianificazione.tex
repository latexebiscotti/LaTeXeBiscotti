\section{Pianificazione}

Al fine della pianificazione si è deciso di suddividere il progetto in quattro macro-fasi distinte:
\begin{itemize}
\item \textbf{Analisi:} incentrata maggiormente sull'analisi dei requisiti
\item \textbf{Seconda fase:}
\item \textbf{Terza fase:}
\item \textbf{Quarta fase:}
\end{itemize}
Ogni macro-fase è stata poi suddivisa in varie attività a loro volta scomposte in sotto-attività ancora più di dettaglio. La dislocazione temporale delle attività è evidenziata attraverso diagrammi di \glossario{Gantt} per rappresentarne la durata, la sequenzialità e il parallelismo. Tali diagrammi includono anche:
\begin{itemize}
\item \textbf{\glossario{Milestone}:} rappresenta la data prevista per la conclusione dell'attività e coincide con la data della consegna dei documenti in vista della revisione associata;
\item \textbf{Attività che compongono la macro-fase:} suddivise poi in sotto attività.
\end{itemize}
Si è scelto di non riportare i diagrammi \glossario{PERT} in quanto, a causa della eccessiva quantità di nodi, si sono rivelati poco leggibili. I grafici \glossario{WBS} evidenziano la struttura gerarchica delle attività: di che attività si compone ogni macro-fase e le sotto-attività di cui ogni attività è composta, tutte univocamente identificate 

\subsection{Analisi}
\textbf{Periodo:} da 2015-12-18 a 2016-01-22 \\
Tale fase inizia in corrispondenza della data di scadenza per la formazione dei gruppi e termina con la data di consegna dei documenti per la revisione dei requisiti. La fase di \textbf{Analisi} prevede la seguente scomposizione in attività:
\begin{itemize}
\item \textbf{Norme di Progetto:} l'\textit{Amministratore} si occupa di redigere ed emanare le norme che tutti i componenti del gruppo dovranno rispettare nello svolgersi dell'intero progetto. Le \textit{Norme di Progetto} sono il primo documento ad essere prodotto in quanto regolano anche la stesura dei documenti e l'utilizzo del software di supporto necessarie già in questa prima fase. Sarà compito dei \textit{Verificatori} assicurare il rispetto di tali norme.
\item \textbf{Studio di Fattibilità:} a seguito della valutazione di tutti i capitolati viene redatto \textit{Studio di Fattibilità}. Si tratta di un documento ad uso interno in cui viene studiata la complessità delle varie proposte. La redazione e approvazione di tale documento condiziona l'inizio dell'\textit{Analisi dei Requisiti}.
\item \textbf{Piano di Progetto:} il \textit{Responsabile} del gruppo è tenuto a redigere in questa fase il \textit{Piano di Progetto} contenente la pianificazione così da organizzare le attività del gruppo.
\item \textbf{Analisi dei Requisiti:} viene generata un'analisi più approfondita, rispetto allo \textit{Studio di Fattibilità}, nell'\textit{Analisi dei Requisiti}. Questa attività continua fino alla data di consegna della documentazione.
\item \textbf{Piano di Qualifica:} \textit{Progettista} e \textit{Verificatore} redigono il \textit{Piano di Qualifica}.
\item \textbf{Lettera di Presentazione:} si tratta di una lettera redatta su carta intestata da presentare al committente che permette al gruppo di partecipare alla gara d'appalto per il capitolato scelto.
\item \textbf{Glossario:} viene scritto in modo incrementale dai redattori dei documenti. Esso contiene la spiegazione di alcuni termini, opportunamente contrassegnati con un G pedice, contenuti all'interno della documentazione. Viene aggiornato passo passo ad ogni aggiunta di termini che necessitano di spiegazione.
\end{itemize}
Nella fase di Analisi i ruoli maggiormente coinvolti sono \textit{Analista}, \textit{Responsabile}, \textit{Amministratore} e \textit{Verificatore}.

\subsubsection{Gantt attività}

\subsubsection{WBS attività}

\subsubsection{Ripartizione ore}

\subsection{Seconda Fase}
\subsubsection{Gantt attività}

\subsubsection{WBS attività}

\subsubsection{Ripartizione ore}

\subsection{Terza Fase}
\subsubsection{Gantt attività}

\subsubsection{WBS attività}

\subsubsection{Ripartizione ore}

\subsection{Quarta Fase}
\subsubsection{Gantt attività}

\subsubsection{WBS attività}

\subsubsection{Ripartizione ore}

