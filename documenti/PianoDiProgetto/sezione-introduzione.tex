\section{Introduzione}

\subsection{Scopo del documento}
Questo documento ha lo scopo di identificare e dettagliare la pianificazione del gruppo \GroupName{} relativamente al progetto \ProjectName{}. 

\subsection{Scopo del Prodotto}
\ScopoDelProdotto

\subsection{Glossario}
Per evitare tutte le possibili ambiguità sul linguaggio utilizzato e per massimizzare la comprensione da parte di tutti dei documenti, della terminologia specifica e di quella di dominio, degli acronimi e di tutte quelle parole che necessitano chiarimento (contraddistinte da una G pedice), viene redatto un \textit{Glossario}, consultabile nel documento \Glossario.

\subsection{Riferimenti}
\subsubsection{Normativi}
\begin{itemize}
\item \textbf{\NormeDiProgetto}
\item\textbf{ Capitolato d'appalto C3:} UMAP: un motore per l'analisi predittiva in ambiente Internet of Things: \url{http://www.math.unipd.it/~tullio/IS-1/2015/Progetto/C3.pdf}
\item \textbf{Vincoli sull'organigramma del gruppo e sull'offerta tecnico-economica:} \url http://www.math.unipd.it/~tullio/IS-1/2015/Progetto/PD01b.html
\end{itemize}

\subsubsection{Informativi}
\begin{itemize}
\item \textbf{Slide dell'insegnamento Ingegneria del Software monulo A:}
\begin{itemize}
\item Ciclo di vita del Software
\item Gestione di Progetto
\end{itemize}
\url{http://www.math.unipd.it/~tullio/IS-1/2015/}
\item \textbf{\textit{Software Engineering} - Ian Sommerville - 9th Edition (2011):}
\begin{itemize}
\item Part4: Software Management
\end{itemize} 
\end{itemize}

\subsection{Ciclo di vita}
Come modello di ciclo di vita si è scelto di adottare il \textbf{modello incrementale} il quale che combina i vantaggi dei modelli a cascata ed evolutivo.

\subsection{Scadenze}

\subsection{Ruoli e costi}
