\section{Introduzione}

\subsection{Scopo del documento}
Questo documento ha lo scopo di identificare e dettagliare la pianificazione del gruppo \GroupName{} relativamente al progetto \ProjectName{}. Si vogliono evidenziare in particolare la ripartizione del carico di lavoro e della responsabilità tra i componenti del gruppo, il prospetto economico preventivo e l'analisi dei rischi.

\subsection{Scopo del Prodotto}
\ScopoDelProdotto

\subsection{Glossario}
Per evitare tutte le possibili ambiguità sul linguaggio utilizzato e per massimizzare la comprensione da parte di tutti dei documenti, della terminologia specifica e di quella di dominio, degli acronimi e di tutte quelle parole che necessitano chiarimento (contraddistinte da una G pedice), viene redatto un \textit{Glossario}, consultabile nel documento \Glossario.

\subsection{Riferimenti}
\subsubsection{Normativi}
\begin{itemize}
\item \textbf{\NormeDiProgetto}
\item\textbf{ Capitolato d'appalto C3:} UMAP: un motore per l'analisi predittiva in ambiente Internet of Things: \url{http://www.math.unipd.it/~tullio/IS-1/2015/Progetto/C3.pdf}
\item \textbf{Vincoli sull'organigramma del gruppo e sull'offerta tecnico-economica:} \url http://www.math.unipd.it/~tullio/IS-1/2015/Progetto/PD01b.html
\end{itemize}

\subsubsection{Informativi}
\begin{itemize}
\item \textbf{Slide dell'insegnamento Ingegneria del Software monulo A:}
\begin{itemize}
\item Ciclo di vita del Software
\item Gestione di Progetto
\end{itemize}
\url{http://www.math.unipd.it/~tullio/IS-1/2015/}
\item \textbf{\textit{Software Engineering} - Ian Sommerville - 9th Edition (2011):}
\begin{itemize}
\item Part4: Software Management
\end{itemize} 
\end{itemize}

\subsection{Ciclo di vita}
Come modello di ciclo di vita si è scelto di adottare il \textbf{modello incrementale} il quale permette di:
\begin{itemize}
\item Soddisfare immediatamente i requisiti principali, e di potersi dedicare successivamente a quelli desiderabili, in modo tale da consegnare subito al cliente un sistema funzionante;
\item Fornire rilasci multipli e successivi ognuno dei quali realizza un incremento di funzionalità;
\item Ridurre il rischio di fallimento del progetto: sebbene possano esserci problemi in alcuni incrementi, è probabile che altri saranno consegnati con successo al cliente;
\item Testare più intensamente i servizi di sistema più importanti in quanto consegnati prima.
\end{itemize}

\subsection{Scadenze}
Di seguito è riportato l'elenco delle scadenze che il gruppo \GroupName{} ha deciso di rispettare e, sulle stesse, si basa la pianificazione presentata in questo documento.
\begin{itemize}
\item \textbf{Revisione dei Requisiti (RR):} 2016-02-16
\item \textbf{Revisione di Progettazione (RP):} 2016-04-18???
\item \textbf{Revisione di Qualifica (RQ):} 2016-05-23
\item \textbf{Revisione di Accettazione (RA):} 2016-06-17
\end{itemize}

Cosa presentiamo alle RQ??? Specifica tecnica o definizione di prodotto??

\subsection{Ruoli}
Durante l'intera attività di progetto i membri del gruppo \GroupName{} sono tenuti a ricoprire, a rotazione, diversi ruoli le cui responsabilità sono descritte nelle \NormeDiProgetto{}. Ogni singolo componente potrà ricoprire più ruoli all'interno della stessa fase del progetto purché sia garantita l'assenza di conflitto di interessi. Si osserva in particolare che una persona non può essere \textit{Verificatore} di se stessa. All'interno della stessa fase è inoltre possibile la duplicazione di ruoli ma devono essere ricoperti da persone distinte. 


