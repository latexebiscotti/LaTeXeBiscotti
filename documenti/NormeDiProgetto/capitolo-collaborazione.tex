\section{Collaborazione}
\subsection{Comunicazioni}
\subsubsection{Comunicazioni Esterne}

Per quanto riguarda le comunicazioni esterne, è stato creato un account Google con associata una casella di posta elettronica: 
\begin{center}
\GroupEmail
\end{center}
Tale indirizzo mail sarà l'unico canale di comunicazione tra il gruppo e l'esterno. A tutti i membri del team sarà permesso l'accesso, ma con l'obbligo di non eliminare mai alcun tipo di mail ricevuta o inviata verso l'esterno. L'unica persona a cui sarà concesso inviare mail verso l'esterno sarà il \textit{Responsabile di Progetto}.

\subsubsection{Comunicazioni Interne}

Per quanto riguarda le comunicazioni interne, sono state pensate più metodologie:

\begin{itemize}
\item \textbf{Forum}: tale forum si trova all'interno dello strumento di \glossario{project management} utilizzato dal gruppo, ovvero \textbf{Teamwork}\footnote[1]{\url{https://www.teamwork.com/}}. Per tutti i membri del team sarà possibile creare nuovi post e dividerli per categorie, in modo tale da strutturare nella maniera migliore possibile i vari argomenti trattati. Usando questo strumento, si avrà uno storico di tutti gli argomenti trattati dal gruppo;
\item \textbf{Messaggistica Istantanea}: è stata creata una chat all'interno della piattaforma \glossario{Telegram}. Con tale mezzo sarà possibile scambiarsi rapidamente dei messaggi urgenti. Se le informazioni trattate dovessero essere di rilievo per lo sviluppo del progetto, i membri sono obbligati a creare un post nel suddetto forum;
\item \textbf{Videoconferenza}: sarà possibile sfruttare il sistema \glossario{Skype}. Verrà quindi, una volta terminata la conversazione, redatto un verbale e inserito sempre come post al già citato forum.
\end{itemize}

\subsubsection{Composizione Post}
In questo paragrafo viene descritta la struttura che deve avere un post da inserire nel forum del gruppo.

\paragraph{Titolo}\mbox{}\\
Il titolo deve essere chiaro ed esaustivo, possibilmente stringato e non confondibile con altri preesistenti. Dovrà inoltre esplicare in modo semplice quale sarà il contenuto del post stesso. Dovranno essere evitati in ogni modo i post riconducibili ad altri all'interno del forum, o post che non hanno significato all'interno della categoria in cui verranno inseriti. In caso di errore, il membro del gruppo sarà tenuto ad eliminare tempestivamente il messaggio, in modo tale da non creare confusione.

\paragraph{Contenuto}\mbox{}\\
Il post dovrà contenere tutte le informazioni necessarie a rendere facilmente comprensibile l'argomento trattato a tutti i membri del gruppo.
Sarà consigliato l'utilizzo di elenchi puntati o numerati per identificare i vari argomenti del post.

\paragraph{Allegati}\mbox{}\\
Sarà possibile inserire degli allegati pertinenti al post, qualora ve ne sia la necessità. Essi possono essere, ad esempio, verbali di una comunicazione via \glossario{instant messaging} o videoconferenza.

\paragraph{Categoria}\mbox{}\\
Tutti i post che andranno inseriti nel forum dovranno appartenere ad una categoria ben definita. Se un componente del gruppo non riesce ad identificarne una per il proprio messaggio, dovrà rivolgersi all'\textit{Amministratore del Progetto}. Sarà compito di quest'ultimo decidere a che categoria associare il messaggio, e, in caso non ne esistesse nessuna consona, spetterà sempre a lui crearne una di nuova. \\ 
\textbf{N.B.}: A nessun membro eccetto l'\textit{Amministratore del Progetto} e il \textit{Responsabile del Progetto} sarà consentito creare nuove categorie all'interno del forum.

\paragraph{Notifiche}\mbox{}\\
Se un messaggio del forum dovesse essere rivolto ad un particolare membro del gruppo, può essergli associata una notifica personale. Se si tratta di un messaggio di comune interesse, ad ogni membro del gruppo dovrà essere associata una notifica.

\subsection{Riunioni}

\subsubsection{Frequenza}

Le riunioni del gruppo di lavoro avranno una frequenza almeno quindicinale.

\subsubsection{Convocazione Riunione}
\paragraph{Riunioni Generali}\mbox{}\\
Il \textit{Responsabile di Progetto} avrà il compito di convocare le riunioni generali, ovvero quelle in cui vengono convocati tutti i membri del gruppo; avrà anche il compito di valutare l'anticipazione della successiva riunione, in caso lo ritenesse necessario. \\
Qualora ve ne sia la necessità, qualsiasi componente del gruppo potrà richiedere la convocazione di una riunione generale; tuttavia, tale richiesta dovrà essere rivolta preventivamente al \textit{Responsabile di Progetto}, il quale deciderà se accogliere o meno la richiesta. \\
Ogni riunione generale dovrà essere convocata, con almeno tre giorni di anticipo, tramite un post nel forum del gruppo, inserito sotto la categoria “Riunioni Generali”.
Tale post dovrà essere così formato:
\begin{itemize}
\item \textbf{Titolo}: Convocazione Riunione n. X, dove X indica il numero progressivo di riunioni generali effettuate;
\item \textbf{Contenuto}: 
\begin{itemize}
\item Data;
\item Orario;
\item Tipo: ordinaria/straordinaria;
\item Ordine del Giorno: elenco numerato delle varie voci da esaminare.
\end{itemize}
\end{itemize}
Ogni membro del gruppo dovrà confermare, o meno, la propria presenza tramite commento al post entro 36 ore. In caso di risposta negativa, dovrà essere fornita una giustificazione. Nell'eventualità che un membro non risponda entro il tempo utile, sarà compito del \textit{Responsabile di Progetto} contattare telefonicamente il membro o i membri del gruppo interessato/i. Una volta ricevuta la risposta da parte di tutti i membri del gruppo, sempre il \textit{Responsabile di Progetto} potrà decidere se confermare, annullare o spostare la riunione, in modo tale da permettere la partecipazione a tutti i componenti del team. In caso decida di spostare la riunione, dovrà commentare nuovamente il post con i dati relativi allo spostamento della riunione. In caso ciò avvenisse, i membri del gruppo sono tenuti a dare una risposta entro 12 ore.

\paragraph{Riunioni tra alcuni membri}\mbox{}\\
Potranno essere necessarie alcune riunioni che andranno a coinvolgere la non totalità dei membri del gruppo. Ad esempio, in fase di Progettazione, è possibile e auspicabile la collaborazione tra \textit{Analista} e \textit{Progettista}: in caso ne sentissero la necessità, potranno indire una riunione che coinvolgerà solamente loro. Per fare ciò, si dovrà scrivere un post nella categoria dedicata (Riunioni tra membri) con la stessa dicitura dei post redatti per le riunioni generali. Per evitare confusione, dovranno essere notificati del post solo i diretti interessati alla riunione. Tutto quello che andrà discusso in tali riunioni dovrà essere verbalizzato, per mettere al corrente di quanto discusso tutti gli altri componenti del gruppo. Il verbale dovrà essere inserito, anche come allegato, ad un post da inserire in categoria “Verbali riunioni membri”.

\paragraph{Svolgimento Riunione}\mbox{}\\
All'apertura della riunione, verificata la presenza dei membri previsti, in caso di riunione tra alcuni dei componenti del gruppo, viene scelto un
segretario che avrà il compito di annotare ogni argomento trattato e di redigere il verbale
dell'assemblea, che dovrà poi essere inserito tramite post nella categoria specifica dei verbali. Tutti i partecipanti devono osservare un comportamento consono al miglior svolgimento
della riunione e al raggiungimento degli obbiettivi della stessa. Il segretario deve inoltre
controllare che venga seguito l'ordine del giorno in modo da non tralasciare alcun punto.

\subsubsection{Verbale}
\paragraph{Riunione Interna}\mbox{}\\
Il verbale di riunione interna è un documento interno informale che consente di tracciare gli argomenti discussi durante la riunione. Sarà scritto dal segretario della riunione, ruolo che andrà scelto di volta in volta, a rotazione, tra i membri del gruppo. Una volta redatto, il verbale dovrà essere inserito come post nell'apposita categoria del forum.

\paragraph{Riunione Esterna}\mbox{}\\
In caso di riunione con il committente od il proponente, il verbale è un documento ufficiale che può avere valore normativo e quindi deve essere redatto seguendo criteri
specifici, specificati nella sezione successiva di questo documento. 

\subsection{Repository e strumenti di condivisione file}
In questa sezione viene descritta la struttura del \glossario{repository} dei documenti. Non è ancora stata definita una struttura per quella del codice. Tale decisione sarà presa nella fase di \textit{Progettazione} dello sviluppo del progetto.
\subsubsection{Struttura Repository dei Documenti}
La struttura del \glossario{repository}, la cui cartella principale viene denominata \texttt{LaTeXeBiscotti}, è così composta:
\begin{itemize}
\item \texttt{LaTeXeBiscotti/documenti/\{NomeDelDocumento\}/} \\ 
Ciascuna cartella descritta da questo percorso contiene i file che vengono utilizzati dal documento \texttt{\{NomeDelDocumento\}}. In particolare conterrà il file \texttt{\{NomeDelDocumento\}.pdf}
e il file \texttt{diario\_modifiche.tex}, contenente il diario delle modifiche del documento;
\item \texttt{LaTeXeBiscotti/ufficiali/} \\
Contiene i documenti ufficiali approvati dal \textit{Responsabile di Progetto};
\item \texttt{LaTeXeBiscotti/modello/} \\
Contiene i file comuni a due o più documenti.
\end{itemize}
È raccomandato che tutti i file e le cartelle non contengano spazi nel loro nome. Non devono mai esserci due file o cartelle il cui percorso differisca soltanto per maiuscole/minuscole. Non bisogna inoltre rinominare file o cartelle modificandone soltanto il \glossario{case} di alcuni caratteri del nome.

\subsubsection{Condivisione dei File}
Per la condivisione informale di file e per il lavoro collaborativo su documenti di supporto, si usa la piattaforma di condivisione documenti prevista da Teamwork. Per una spiegazione più specifica si rimanda a §6.1.2.
